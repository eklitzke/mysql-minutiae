\documentclass[14pt]{beamer}
\usepackage{hyperref}
\usetheme{CambridgeUS}

\title{MySQL Minutiae \& InnoDB Internals}
\author{Evan Klitzke \and Daniel Chen}
\institute{Yelp, Inc.}
\date{March 18, 2011}

\begin{document}

\begin{frame}
  \titlepage
\end{frame}

\begin{frame}{A Huge Disclaimer}
  We're not really MySQL or InnoDB experts.
  \newline
  \newline
  This presentation probably contains gross inaccuracies, especially regarding
  InnoDB internals.
  \newline
  \newline
  We're going to try to go really fast so try to save questions for the end,
  unless we're being totally incoherent or you think we're making an egregious
  error.
\end {frame}

\begin{frame}{Outline}
  \tableofcontents
\end{frame}

\section{The Basics}

\begin{frame}{Introduction}
  A DBA walks into a bar, and sees two tables...
  \pause
  \newline
  he walks up to them and asks, ``May I join you?''
\end {frame}

\begin{frame}{Introduction}
  Today we'll be covering the MySQL relational database at a high-level, and
  diving more deeply into a particular storage engine of MySQL, the InnoDB
  storage engine.
  \newline
  \newline
  A bit of context: mostly at Yelp we're on MySQL 5.0 (5.1 is coming \emph{real
    soon}), and we're almost exclusively using InnoDB as our storage
  engine. Almost all data that's rendered on any page load on our site comes out
  of MySQL.
\end {frame}

\begin{frame}{Databases}
  Things a relational database does:
  \begin{itemize}
    \item stores data
    \item provides convenient query facilities (i.e. SQL)
    \item hopefully: provides transaction support (and transaction isolation)
    \item maybe: replication facilities
  \end{itemize}
  MySQL with the InnoDB storage engine does all of these.
\end{frame}

\begin{frame}{Database Transactions}
  Database transactions are an \emph{extremely} important concept, and will be
  pivotal to our discussion.
  \pause
  \newline
  \newline
  A \emph{database transaction} provides the capability for connections to atomically execute groups of statements.
\end{frame}

\begin{frame}[fragile]
  \frametitle{Database Transaction Example 1}
  Consider the following query
  \begin{verbatim}
BEGIN
INSERT INTO users (...)
INSERT INTO users (...)
COMMIT
  \end{verbatim}

  The transaction ensures that either both inserts occur, or neither insert
  occurs. There's no way for just one of the rows to be inserted.
\end{frame}

\begin{frame}[fragile]
  \frametitle{Database Transaction Example 2}
  What is returned at (a), (b), and (c) in REPEATABLE READ?
  \begin{verbatim}
T1: BEGIN
T2: BEGIN
T3: BEGIN
T1: INSERT INTO t VALUES (1)
T2: SELECT * FROM t // (a)
T1: COMMIT
T2: SELECT * FROM t // (b)
T3: SELECT * FROM t // (c)
  \end{verbatim}
\end{frame}

\begin{frame}{Database Transaction Example 2}
(a) Returns no rows
\newline
(b) Returns no rows
\newline
(c) Returns one row, [1]
\newline
Takeaway: Transactions actually begin when you issue your first query
\end{frame}

\begin{frame}{Transaction Isolation Levels}
  The SQL specification provides for four different \emph{transaction isolation
    levels}. These control how much isolation there is between two transactions.
  \newline
  \newline
  In general you can choose where you want to be on a scale of low
  isolation/overhead and high isolation/overhead. The isolation level you're in
  has important ramifications when it comes to locking, as we'll see later.
\end{frame}

\begin{frame}{READ UNCOMMITTED}
  The lowest isolation level is \texttt{READ UNCOMMITTED}; any change from any
  transaction can be read in this transaction isolation level, even if the
  change has not been committed.
  \newline
  \newline
  Don't ever use this.
\end{frame}

\begin{frame}{READ COMMITTED}
  In \texttt{READ COMMITTED} you can read any change committed by any other
  transaction. This has low overhead, and is frequently ``good enough''.
  \newline
  \newline
  In practice this is the least strict isolation level implemented by most
  databases (e.g. PostgreSQL and Oracle).
\end{frame}

\begin{frame}{REPEATABLE READ}
  In \texttt{REPEATABLE READ}, a consistent snapshot of the database is
  available to the transaction. If you execute \texttt{SELECT * FROM user WHERE
    ...}, that query will always return the same set of rows within the
  transaction. Deletes, updates, and inserts by other transactions will not
  affect what you read, even after those transactions have been comitted.
  \newline
  \newline
  This is the default isolation level in MySQL, and the one primarily used by Yelp.
\end{frame}

\begin{frame}{REPEATABLE READ (continued)}
  Note that in \texttt{REPEATABLE READ} MySQL must maintain a snapshot of what
  the database looked like when the transaction began (this is called
  MVCC, multi-version concurrency control).
  \newline
  \newline
  As rows are inserted/updated/deleted, MySQL must maintain state to remember
  what the snapshot should look like to each transaction executing in
  \texttt{REPEATABLE READ}.
\end{frame}

\begin{frame}{REPEATABLE READ (continued)}
  Even \texttt{SELECT} statements incur overhead in \texttt{REPEATABLE READ},
  and can cause locking problems and database overhead.
  \newline
  \newline
  You must \emph{never} leave a transaction open indefinitely after doing any
  SQL query in \texttt{REPEATABLE READ}. Periodically \texttt{COMMIT} or
  \texttt{ROLLBACK}, even when you're just doing selects.
\end{frame}

\begin{frame}{REPEATABLE READ (continued)}
  This is so important it will be repeated. The official documentation says:
  \newline
  \newline
  ``You must remember to commit your transactions regularly, including those
  transactions that issue only consistent reads. Otherwise, InnoDB cannot
  discard data from the update undo logs, and the rollback segment may grow too
  big, filling up your tablespace.''
\end{frame}

\begin{frame}{SERIALIZABLE}
  In \texttt{SERIALIZABLE} the database must act as if only one transaction
  occurred at a time, i.e. as if the database itself were single-threaded. In
  practice, this is implemented by causing all selects to automatically acquire
  S-locks.
  \newline
  \newline
  In you need to use this, your code is broken. It's better to acquire the locks
  explicitly using \texttt{LOCK IN SHARE MODE} or \texttt{SELECT FOR UPDATE}.

\end{frame}


\begin{frame}{MySQL Storage Engines}
  MySQL has the concept of a ``storage engine''. This is an implementation of
  facilities for storing data and dealing with things like database transactions
  and locking.
  \newline
  \newline
  The most commonly used storage engines are MyISAM and InnoDB. The storage
  engine is set on a per-table level, so any given table can use any storage
  engine.
\end{frame}

\begin{frame}{MySQL Storage Engines}
  To better elucidate the diferences, MySQL itself does the following:
  \begin{itemize}
    \item parses incoming SQL statements, and does some of the query optimization
    \item handles replication
  \end{itemize}
  The storage engine
  \begin{itemize}
    \item implements on-disk storage, and caching
    \item implements transactions and transaction isolation
  \end{itemize}
\end{frame}

\begin{frame}{MySQL Storage Engines}
  At Yelp, we
  almost exclusively use the InnoDB storage engine. Compared to MyISAM, InnoDB provides:
  \begin{itemize}
    \item ``row-level'' locking, instead of table-level locking
    \item replication safety
    \item better crash recovery
  \end{itemize}
  In general InnoDB is more actively maintained than MyISAM, and is usually
  comparable or superior in speed.
\end{frame}

\section{Replication}

\begin{frame}{MySQL Replication}

  MySQL has this great feature called \emph{replication}.
  \newline
  \newline
  When replication is
  enabled database inserts, updates, and deletes can be propagated from a \emph{master} database
  instance to one or more \emph{slaves}.
\end{frame}

\begin{frame}{MySQL Replication}
  Typically that queries that modify data---statements like \texttt{INSERT},
  \texttt{UPDATE}, \texttt{DELETE}---should only be run on the master
  database. Running these statements on a slave will cause it to get out of sync
  with the master, which is \emph{usually} unintentional.
  \newline
  \newline
  Queries that just read data (i.e. \texttt{SELECT} statements) can be run on any database, master or slave.
\end{frame}

\begin{frame}{Uses of MySQL Replication}
  What is replication good for?
\end{frame}

\begin{frame}{Uses of MySQL Replication}
  Replication helps increase reliability. Your data is stored in more than one
  place, so if the master MySQL instance fails, the data exists on one or more slaves.
  \newline
  \newline
  In the event of a failure, you could choose to promote a slave to become a
  master, reducing site downtime.
\end{frame}

\begin{frame}{Uses of MySQL Replication}

  Replication makes it easy to scale out read capacity.
  \newline
  \newline
  \texttt{SELECT}
  statements can be run against slave databases. As traffic grows you can
  purchase more servers to use as slaves, instead of having to purchase a huge
  mainframe database to handle all traffic.

\end{frame}

\begin{frame}{Uses of MySQL Replication}
  Replication can assist in data recovery, and testing out changes. It's
  possible to test things like \texttt{ALTER} statements on a slave without
  affecting the master.
  \newline
  \newline
  Strategies like delayed replication can be used to help
  prevent certain types of data corruption.
\end{frame}

\begin{frame}{MySQL Binlogs}
  MySQL serializes transactions that modify data into a special log called the
  \emph{binlog}, short for ``binary log''.
  \newline
  \newline
  SQL statements in the binlog are replayed one at a time on the slave
  databases, causing them to make the same modifications that the master made.
\end{frame}

\begin{frame}{Replication Caveats}
  The most common replication strategy used in MySQL, and the only option in
  MySQL 5.0, is \emph{statement-based replication}. This means SQL statements
  are copied verbatim into the binlog, and re-run on the slaves.
  \newline
  \newline
  An update statement with a where clause might execute slowly (e.g. minutes),
  and end up modifying just a single row. This statement will take just as long
  to run on the slaves.
\end{frame}

\begin{frame}{Replication Caveats}
  In passing, we note that MySQL 5.1 and later supports \emph{row-based
    replication}, which copies that actual changes made by the master into the
  binlog, rather than SQL queries.
  \newline
  \newline
  In general row-based replication is still not that mature, and has a lot of
  corner cases. We don't use it, and it's disabled in MySQL by default.
\end{frame}

\begin{frame}{Replication Caveats}
  When a slave replays statements in the binlog, it replays the statements one
  at a time (i.e. from a single thread). In contrast, the master database can
  run statements concurrently in multiple threads.
\end{frame}

\begin{frame}{Replication Caveats}
  Because the binlog is implicitly single threaded, it's easy for slaves to fall
  behind in processing binlog statements, even though they have the same
  hardware as the master, and the master is easily able to keep up.
  \newline
  \newline
  This is very important: just because the master database can keep up with your
  queries does not mean that the slaves will.
\end{frame}

\begin{frame}{Replication Delay}
  When slaves cannot replay replication statements as quickly as the master,
  their copy of the data falls behind and gets out of date.
  \newline
  \newline
  A query on the master might return different data than the same query on a
  slave, since the slave as an old copy of the data.
\end{frame}

\begin{frame}{Replication Delay}
  In the Yelp code there are tricks like cacheserv and the ``dirty session''
  cookie that generally make it impossible to observe replication delay under
  normal circumstances, even if the slaves are moderately (e.g. $\le 2$ minutes)
  delayed.
  \newline
  \newline
  But you should still be careful!
\end{frame}

\section{HandlerSocket}

\begin{frame}{Handler API}
  The Handler API is the API you implement if you want to write your own MySQL
  storage engine. You can implement it at two levels, a ``simple'' version for
  an ultra-minimal ISAM like storage engine, or an ``advanced'' version for
  something like InnoDB that supports transactions, two-phase commit, etc.
  \newline
  \newline
  \pause
  You'll see reference to Handler in storage engine names such as
  \texttt{ha\_innodb}, \texttt{ha\_ndbcluster}, etc.
\end{frame}

\begin{frame}{Handler API}
  According to cryptic MySQL slides on the internet, you only need to implement
  ``about 10'' functions to implement the simplest level of compliance, ``and
  half of them are one-liners''.
\end{frame}

\begin{frame}[fragile]
  \frametitle{Handler Variables}
  For a good time, run the following MySQL query:
\begin{verbatim}
SHOW GLOBAL STATUS LIKE 'Handler_%'
\end{verbatim}
  You'll see about fifteen handler-related global variables about total number
  of rows read, deleted, updated, etc.
\end{frame}

\begin{frame}{HandlerSocket}
  HandlerSocket is a project, implemented as a MySQL plugin, that takes
  advantage of the handler API to provide fast low-level access to MySQL storage
  engines.
\end{frame}

\begin{frame}{HandlerSocket}
  HandlerSocket creates a new thread inside of MySQL, which creates a TCP socket
  bound to some port.
  \newline
  \newline
  You connect to the socket and interact with the HandlerSocket plugin using a
  compact binary network protocol.
\end{frame}

\begin{frame}{HandlerSocket}
  The HandlerSocket API exposes the basic Handler API functions for
  reading/writing rows, doing updates, etc.
  \newline
  \newline
  Basically you can do very simple CRUD (create-read-update-delete) operations
  without the overhead of SQL parsing.
\end{frame}

\begin{frame}{HandlerSocket}
  Another way to think of it: HandlerSocket lets you use an engine like InnoDB
  without all of the overhead and complexity of the rest of MySQL.
\end{frame}

\begin{frame}{HandlerSocket Performance}
  Akira Higuchi (the HandlerSocket author) has generally found about
  $4\times$~improvement for PK-read queries.
  \newline
  \newline
  Likewise, generally about $3\times$~improvement for inserts.
\end{frame}

\begin{frame}{HandlerSocket Benchmark}
  These are the results of Yoshinori Matsunobu's benchmarks, using the InnoDB
  plugin version 1.0 (i.e. \texttt{ha\_innodb\_plugin.so} for MySQL 5.1):
  \begin{table}[ht]
    \begin{tabular}{l l r r}
                              & QPS     & \%us & \%sy \\ \hline
      MySQL via SQL           & 105,000 & 60   & 28   \\
      memcached               & 420,000 & 8    & 88   \\
      MySQL via HandlerSocket & 750,000 & 45   & 53   \\
    \end{tabular}
  \end{table}
\end{frame}

\begin{frame}{HandlerSocket Benchmark}
  My benchmark. This is with a large InnoDB buffer pool (large enough to
  accomodate all data), and with \texttt{fsync()} disabled after commits. All
  tests are for 100,000 reads/writes and I took 20 samples. The key was always
  an integer, as was the value.

  \begin{tabular}{l r r r r r}
                      & min   & Q1    & Q2    & Q3    & max   \\ \hline
    HS writes         & 16.06 & 17.78 & 18.43 & 19.43 & 20.30 \\
    HS reads          & 13.08 & 13.22 & 13.31 & 13.37 & 13.49 \\
    MC writes         &  9.95 & 10.72 & 10.99 & 11.08 & 11.23 \\
    MC reads          & 10.29 & 10.58 & 10.80 & 11.09 & 11.46 \\
  \end{tabular}
\end{frame}

\begin{frame}{HandlerSocket Benchmark}
  Same, but with 20-byte keys and 500-byte values.
  \begin{tabular}{l r r r r r}
                      & min   & Q1     & Q2     & Q3     & max    \\ \hline
    HS writes         & 84.04 & 109.90 & 110.80 & 116.00 & 124.30 \\
    HS reads          & 14.45 & 14.63  & 14.69  & 14.92  & 15.48  \\
    MC writes         & 12.59 & 12.84  & 13.00  & 13.14  & 13.19  \\
    MC reads          & 11.23 & 12.00  & 12.11  & 12.44  & 12.90  \\
  \end{tabular}
\end{frame}

\begin{frame}{HandlerSocket Benchmark}
  And here's one more benchmark, that I did on my home laptop with MySQL 5.1.55,
  a consumer-grade Intel SSD, btrfs, and
  \texttt{innodb\_flush\_log\_at\_trx\_commit = 1}. As before, 20-byte keys and
  500-byte values.

  \begin{table}[ht]
  \begin{tabular}{l r r r}
           & MC    & HS     & SQL \\ \hline
    writes & 12.57 & 494.70 & 496.82 \\
    reads  & 8.52  & 22.84  & 1864.16\\
  \end{tabular}
  \end{table}
  (note: these numbers are very suspicious, so don't draw any conclusions)
\end{frame}

\begin{frame}{HandlerSocket Conclusions}
  The performance that I (Evan) measured doesn't quite match what others have
  reported, so more investigation here is warranted, probably using real DB
  hardware.
  \newline
  \newline
  In general, it looks like HandlerSocket read performance is nearly as good as
  memcached, and could make a good replacement for row-level caches (e.g. \texttt{user\_session} data).
\end{frame}

\section{InnoDB Internals}

\begin{frame}{InnoDB Background}
  InnoDB is pretty old; first line of code written in early 1994, first released
  with MySQL-3.23.34a in 2001.
  \newline
  \newline
  It was primarily written by Heikki Tuuri and other Finns of \emph{Innobase Oy}
  who wrote cryptic English documentation, and even more cryptic C++ code.
  \newline
  \newline
  A lot of people underestimate how amazingly fast and advanced InnoDB
  is. There's black magic \emph{everywhere}.
\end{frame}

\begin{frame}{InnoDB Structure}
  InnoDB consists of several various substructures such as the buffer pool, log
  files, tablespace, etc.
\end{frame}

\begin{frame}{InnoDB On-Disk Row Structure}
  In general, data is stored inside of a row, instead of as a pointer (but very
  large \texttt{TEXT} and \texttt{BLOB} columns, generally $>8000$ bytes, may be
  different).
  \newline
  \newline
  Up to 50\% space may be wasted due to alignment issues in the most
  pathological cases, but generally InnoDB is conservative about using space.
\end{frame}

\begin{frame}{InnoDB On-Disk Row Structure}
  Three fields are automatically added to all rows:
  \begin{table}[ht]
    \begin{tabular}{l l l}
      field                  & bytes & purpose \\ \hline
      \texttt{DB\_TRX\_ID}   & 6     & transaction id \\
      \texttt{DB\_ROLL\_PTR} & 7     & ``roll pointer'' to undo log records \\
      \texttt{DB\_ROW\_ID}   & 6     & auto-increment value
    \end{tabular}
  \end{table}
\end{frame}

\begin{frame}{The InnoDB Buffer Pool}
  InnoDB keeps an LRU copy of table and index data in memory, in a special
  structure called the \emph{buffer pool}. When queries are executed, InnoDB
  will try to fulfil as much of the query as possible by consulting the buffer
  pool, instead of seeking to disk.
  \newline
  \newline
  Memory is \emph{much} faster than disk.
\end{frame}

\begin{frame}[fragile]
  \frametitle{The InnoDB Buffer Pool}
  Assuming average memory and a normal HDD:
  \begin{table}[ht]
    \begin{tabular}{l r}
      operation                    & nanoseconds \\ \hline
      L1 cache reference           & $<1$        \\
      L2 cache reference           & 7           \\
      main memory (RAM) reference  & 100         \\
      flash (FusionIO) reference   & 50,000      \\
      read 1MB sequentially (RAM)  & 250,000     \\
      disk seek                    & 10,000,000  \\
      read 1MB sequentially (disk) & 30,000,000  \\
    \end{tabular}
  \end{table}
\end{frame}

\begin{frame}{The InnoDB Buffer Pool}
  Note that InnoDB does \emph{not} try to make use of the kernel page cache
  (i.e. the Linux ``disk cache''). InnoDB can structure data more efficiently
  itself, allocate optimally sized pages, \texttt{mlock(2)}/\texttt{madvise(2)}
  the memory, etc.
  \newline
  \newline
  Typically InnoDB operates using \texttt{O\_DIRECT}, meaning that all access to
  InnoDB data files completely bypasses the Linux VFS page cache.
\end{frame}

\begin{frame}{InnoDB Log Files}
  To ensure that data corruption errors are recoverable in the case of a crash,
  InnoDB uses a standard two-phase commit system, updating a log file before
  updating the actual tablespace.
  \newline
  \newline
  This is basically the same as how journalling filesystems like ext4 and HFS+ work.
\end{frame}

\begin{frame}{InnoDB LSN}
  All queries update a globally incrementing log sequence number, the LSN. The
  ``current'' LSN is stored alongside row data, to provide row versioning.
  \newline
  \newline
  There's also a pointer to the previous copy of each row, which is used in
  InnoDB's MVCC implementation.
\end{frame}

\begin{frame}{InnoDB LSN}
  MySQL can safely purge any old rows that have been deleted/updated if their
  LSN is lower than the lowest LSN of any open transaction.
  \newline
  \newline
  If you forget to commit/rollback after a read, the minimum LSN won't decrease,
  and old rows can't be reclaimed.
\end{frame}

\begin{frame}{InnoDB LSN}
  MySQL can use a transaction's LSN to figure out if it's OK for a row's data to
  be seen by a transaction, or if it must find an older version of the
  row.
  \pause
  \newline
  \newline
  e.g. this txn has LSN 100, and this row has LSN 120, ignore this row.
  \pause
  \newline
  \newline
  e.g. this txn has LSN 100, this row has LSN 120 and \texttt{DB\_ROLL\_PTR} to
  LSN 90 in, follow the pointer.
\end{frame}

\begin{frame}{InnoDB Tablespace}
  InnoDB uses the term \emph{tablespace} to refer to the actual on-disk storage
  of tables and indexes. It sounds awesome but it's just jargon, nothing
  interesting to see here.
  \newline
  \newline
  If you're a DBA, you'll know the tablespace as the file \texttt{ibdata1} and
  associated \texttt{.ibd} and \texttt{.frm} files.
\end{frame}

\section{InnoDB Locking}

\begin{frame}{InnoDB Locking}
  There are three important types of locks in InnoDB. They are:
  \begin{itemize}
    \item Auto-increment locks
    \item Table locks
    \item Index locks
  \end{itemize}
  We will present them in this order, from least complext to most complex. Note
  that some of these have sub-varieties of locks.
\end{frame}

\begin{frame}{Lock Acquisition Order}
  Generally a single query will acquire locks in this order:
  \begin{enumerate}
    \item Table locks
    \item Auto-increment locks (inserts only)
    \item Index locks
  \end{enumerate}
\end{frame}

\subsection{Auto-Increment Locks}

\begin{frame}{Auto-Increment Locks}
  Auto-increment locks are special and different from all other locks. They're
  also the simplest, so we'll cover them first.
\end{frame}

\begin{frame}[fragile]
  \frametitle{Auto-Increment Locks}
  Commonly you'll see tables with a column defined like
\begin{verbatim}
id INTEGER PRIMARY KEY AUTO_INCREMENT
\end{verbatim}
  If any column on a table is declared as \texttt{AUTO\_INCREMENT}, the table
  will have an auto-increment lock associated with it. There can be at most one
  auto-increment column (and lock) per table.
\end{frame}

\begin{frame}{Auto-Increment Locks}
  Auto-increment locks are acquired by statements, not transactions (unlike
  every other lock).
  \newline
  \newline
  The lock is held for the duration of a single \texttt{INSERT} statement and
  then immediately released, even if there are multiple inserts in a single
  transaction.
\end{frame}

\begin{frame}{Auto-Increment Locks}
  Upon executing an \texttt{INSERT} into a table with an auto-increment lock,
  the auto-increment lock is acquired after table locks and before index
  locks.
\end{frame}

\begin{frame}{Auto-Increment Locks}
  Note that the auto-increment value is increased even if a statement that
  increases it isn't committed, so you can have ``gaps'' in a table even if
  there were no deletions.
  \newline
  \newline
  \pause
  Likewise, the auto-increment values created by multiple inserts in a
  transaction may not be contiguous.
\end{frame}

\begin{frame}{Auto-Increment Locks}
  In passing, we note that MySQL 5.1 has improved the performance of
  auto-increment locks (compared to 5.0), so this is something to look forward
  to!
\end{frame}

\subsection{Lock Types}

\begin{frame}{Lock Types}
  Before diving into table and index locks, we have to discuss lock types. In
  general there are two types of locks. There are \emph{shared locks},
  a.k.a. ``read'' locks or S-locks.
  \newline
  \newline
  There are also \emph{exclusive locks}, a.k.a. ``write'' locks or X-locks.
  \newline
  \newline
  \pause
  Be advised: we'll be using the notation S-lock and X-lock a lot.
\end{frame}

\begin{frame}{Lock Types (S-locks)}
  S-locks are generally used when values are read, and prevent other
  transactions from modifying the data. Multiple transactions can have S-locks
  on the same data simultaneously.
  \newline
  \newline
  S-locks are weaker than X-locks, and may sometimes be upgraded to
  an exclusive lock later on.
\end{frame}

\begin{frame}{Lock Types (X-locks)}
  X-locks are stronger than S-locks, and are incompatible with other locks. If
  transaction $A$ holds an X-lock on some data, no other transaction can hold an
  S-lock \emph{or} an X-lock on that data at the same time; the other
  transactions must wait until $A$ releases the lock.
  \newline
  \newline
  If other transactions already hold S-locks or X-locks, then $A$ must wait
  until those transactions release their lock before it can acquire the X-lock.
\end{frame}

\begin{frame}{Lock Types}
  Just to reiterate: the general rule of thumb is that S-locks might be acquired
  by reads, and X-locks are always acquired by writes. S-locks are compatible
  with other S-locks, and X-locks are compatible with nothing.
\end{frame}

\subsection{Table Locks}

\begin{frame}{Table Locks}
  Tables can be locked in InnoDB. When a table is locked with an S-lock no
  inserts/updates/alters can occur, but reads can.
  \newline
  \newline
  When a table is locked with an X-lock no other queries can read or write to
  that table.
\end{frame}

\begin{frame}{Table Locks}
  The most common reason for an S-lock or X-lock to be acquired on a table is
  because of an administrative command such as an \texttt{ALTER TABLE} or
  \texttt{DROP TABLE}.
  \newline
  \newline
  In ``normal'' usage you won't see these locks.
\end{frame}

\begin{frame}{Intent Locks}
  Tables also have special ``intent'' locks, IS-locks (``intent to acquire
  S-lock'') and IX-locks (``intent to acquire X-lock'').
  \newline
  \newline
  Intent locks are always compatible with other intent locks (e.g. IX-lock is
  compatible with IS-lock).
\end{frame}

\begin{frame}{Intent Locks}
  Here is a handy matrix showing lock compatibility ($\times$~means
  incompatible):
  \begin{table}[ht]
    \begin{tabular}{r|c c c c}
        & IS       & S        & IX       & X        \\ \hline
     IS & $\cdot$  & $\cdot$  & $\cdot$  & $\times$ \\
     S  & $\cdot$  & $\cdot$  & $\times$ & $\times$ \\
     IX & $\cdot$  & $\times$ & $\cdot$  & $\times$ \\
     X  & $\times$ & $\times$ & $\times$ & $\times$ \\
    \end{tabular}
  \end{table}
\end{frame}

\begin{frame}{Intent Locks}
  Any query that's going to acquire an S-lock or X-lock on an index will acquire
  an intent lock of the same type on the table first.
  \newline
  \newline
  This is so that queries
  like \texttt{ALTER TABLE} are prevented from running while a \texttt{SELECT}
  or \texttt{INSERT} is running on a table, even though those queries don't
  actually lock the table.
\end{frame}

\subsection{Index Locks}

\begin{frame}{Index Locks}
  Take a deep breath, we're about to enter the magic and mysterious world of index locks.
  \newline
  \newline
  These are by far the most complex, and arguably most important type of lock in InnoDB.
\end{frame}

\begin{frame}{Index Locks}
  Also, remember to forget everything we just said about intent locks. Those are
  for table locks only, not index locks.
  \newline
  \newline
  Index locks come in just two types: S-locks and X-locks.
\end{frame}

\begin{frame}{Index Locks}
  Index locks \emph{are} just of the kind S or X, but that's not the whole
  story. There are three flavors of index lock:
  \begin{itemize}
    \item index record locks
    \item next-key locks
    \item gap locks
  \end{itemize}
  What all of these locks have in common is that they are held at a per-index
  level granularity, and they lock all or part of an index.
\end{frame}

\begin{frame}{Index Record Locks}
  The most basic index lock is the \emph{index record lock}. Some people may
  confuse row-level locks with these. In fact, InnoDB cannot truly lock rows, it
  instead locks these index records.
  \newline
  \newline
  When no indexes/keys are defined for a table, an implicit auto-increment lock
  is created, and that index is used for locking records. Index record locks on
  the PK effectively act as a row lock.
\end{frame}

\begin{frame}[fragile]
  \frametitle{Index Record Locks}
  Consider the query
\begin{verbatim}
DELETE FROM user WHERE id = 1
\end{verbatim}
  this query will lock the part of the index for the row whose \texttt{id} is
  1. The same goes for \texttt{UPDATE} statements, etc.
  \newline
  \newline
  An index record lock is acquired for all indexes associated with the table
  (with the potential to deadlock, lock wait timeout, etc.). This is one reason
  not to create too many locks on a table.
\end{frame}

\section{Examples}

\begin{frame}[fragile]
  \frametitle{Next-Key Locking}
\begin{verbatim}
CREATE TABLE dchen (i INT PRIMARY KEY);
INSERT INTO dchen (i) VALUES (10), (20), (30);
BEGIN;
SELECT * FROM dchen WHERE i > 15;
\end{verbatim}
  How many locks does the transaction hold, and what are they?
\end{frame}

\begin{frame}[fragile]
  \frametitle{Copy Row Example}
  The goal is to write a function \texttt{copy\_row(id\_src, id\_dst)} that
  copies the values from the row \texttt{id\_src} to \texttt{id\_dst}. The
  pseudo-code is like:
\begin{verbatim}
  BEGIN;
  SELECT * FROM table WHERE id = id_src;
  UPDATE table SET ... = ... WHERE id = id_dst;
  COMMIT;
\end{verbatim}
  If there are two simultanous invocations like \texttt{copy\_row(1, 2)} and
  \texttt{copy\_row(2, 1)}, then both rows should end up equal but not swapped.
\end{frame}

\begin{frame}{Copy Row Example}
  Question is: what locks must be acquired to ensure that rows always end up
  copied, and not swapped?
  \newline
  \newline
  Which uses of \texttt{LOCK IN SHARE MODE} or
  \texttt{SELECT FOR UPDATE}, or which isolation levels, are sufficient to
  guarantee this?
\end{frame}

\begin{frame}{LOCK IN SHARE MODE Example}
  Table \texttt{stupid} has one column \texttt{id INT PRIMARY KEY} and one row
  whose value is 10.
  \newline
  \newline
  $A$: \texttt{SELECT * FROM stupid WHERE id > 5 LOCK IN SHARE MODE;}
  \newline
  \newline
  $B$: \texttt{INSERT INTO stupid (id) VALUES (1);}
  \newline
  \newline
  What happens?
\end{frame}


\begin{frame}{FOR UPDATE example 1}
  $A$: \texttt{SELECT * FROM user WHERE id = 1 FOR UPDATE;}
  \newline
  \newline
  $B$: \texttt{SELECT * FROM user WHERE id = 1;}
  \newline
  \newline
  Will $B$ block on $A$'s X-lock?
\end{frame}

\begin{frame}{FOR UPDATE example 2}
  $A$: \texttt{SELECT * FROM user WHERE id = 1 FOR UPDATE;}
  \newline
  \newline
  $B$: \texttt{SELECT * FROM user WHERE id = 1 LOCK IN SHARE MODE;}
  \newline
  \newline
  Will $B$ block on $A$'s X-lock?
\end{frame}

\begin{frame}{Insert Deadlock}
  Users with ids 1, 3, 4, 7, 8 exist in the table at the start.
  \newline
  \newline
  $A$: \texttt{DELETE FROM user WHERE id = 5;}
  \newline
  $B$: \texttt{DELETE FROM user WHERE id = 6;}
  \newline
  $A$: \texttt{INSERT INTO user (id) VALUES (5); }
  \newline
  $B$: \texttt{INSERT INTO user (id) VALUES (6); }
  \newline
  \newline
  Why does this deadlock?
  \newline
  Bonus: is $A$ killed or $B$, and why?
\end{frame}
\begin{frame}{Conclusion}
  The source code and compiled PDF of these slides is available online at
  \newline
  \url{https://github.com/eklitzke/mysql-minutiae}
  \newline
  \newline
  See also my Percona Live notes at
  \newline
  \url{https://gist.github.com/830703}
\end{frame}

\end{document}
